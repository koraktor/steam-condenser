\documentclass[draft=false, fontsize=12pt, oneside, paper=a4, DIV15,
twocolumn=false, footsepline, headsepline, titlepage, parskip=half]{scrreprt}

\usepackage{color}
\usepackage[utf8x]{inputenc}
\usepackage[T1]{fontenc}
\newcommand{\changefont}[3]{\fontfamily{#1} \fontseries{#2} \fontshape{#3} \selectfont}
\usepackage[breaklinks]{hyperref}
\usepackage{listings}

\definecolor{lightblue}{rgb}{0.9, 0.9, 1.0}
\definecolor{lightgreengray}{rgb}{0.8, 0.95, 0.8}

\lstset{
    basicstyle=\tt\small, breaklines=true, breakatwhitespace=true,
    keywordstyle=\bfseries\color{blue},
    stringstyle=\itshape, showstringspaces=false, numbers=left,
    frame=single, backgroundcolor=\color{lightgreengray}}

\newcommand{\steamcondenser}{\textbf{Steam Condenser}}
\newcommand{\note}[1]{
	\begin{center}
	\colorbox{lightblue}{
	\begin{minipage}{0.8\textwidth}
	\rule[1ex]{\textwidth}{1mm}
	\textbf{Note:}\\#1\\
	\rule[1ex]{\textwidth}{1mm}
	\end{minipage}
	}
	\end{center}
}

\begin{document}

\changefont{cmss}{m}{n}
\addtokomafont{subsection}{\changefont{cmss}{bx}{sl}}
\addtokomafont{footnote}{\changefont{cmss}{m}{n}}

\author{Sebastian Staudt}
\title{
    \huge{Steam Condenser}\\
    \small{- General Documentation -}\\
    \large{Version 0.7}
}
\date{\today}

\maketitle


\tableofcontents

\chapter{Introduction}
\steamcondenser\ is a library to acquire information from Valve's online
platform Steam. This includes querying game servers based on Valve's game
engines GoldSrc and Source, querying the Steam master servers to find game
servers and gathering user information from the Steam
Community\footnotemark[1]. At the moment the library is implemented in Java,
PHP and Ruby.

\footnotetext[1]{The Steam Community feature is usable but incomplete as it's
based on Valve's XML data which is still work-in-progress.}

\chapter{Using \steamcondenser}

\section{Server queries}

\subsection{Querying game servers}
To query game servers (both GoldSrc and Source engine based), you have to create
a corresponding server object first (\lstinline{GoldSrcServer} or
\lstinline{SourceServer}). After that you should use the method
\lstinline{initialize()} to query basic information from the server. This
includes measuring latency (Ping), receiving a challenge number and other
querying other basic information like the name of the server, number of players
and the map running at the moment. After that you may use the methods
\lstinline{getPlayers()} and \lstinline{getRules()} to query the server for
additional information about the players on the server and the server rules
respectively.

\begin{lstlisting}[caption=Querying a local Source game server listening on port
	27016 in Ruby, language=Ruby]
	server = SourceServer.new "127.0.0.1", 27016
	server.initialize
	server.get_players
	server.get_rules
	...
\end{lstlisting}

\subsection{Querying master servers}
Querying Steam's master servers is done with the help of the
\lstinline{MasterServer} object. As the master server protocol for both
GoldSrc and Source is the same, only the address of the master servers
differ. These are stored in class constants of the class
\lstinline{MasterServer}, namingly \lstinline{GOLDSRC_MASTER_SERVER} for
GoldSrc and \lstinline{SOURCE_MASTER_SERVER} for Source respectively.

\begin{lstlisting}[caption=Querying the GoldSrc master server and getting a
	random GoldSrc server in Java, language=Java]
	MasterServer master = new MasterServer(MasterServer.GOLDSRC_MASTER_SERVER);
	Vector<InetSocketAddress> server = master.getServers();
	InetSocketAddress randomServer = servers.elementAt(randomizer.nextInt(servers.size()));
	GoldSrcServer server = new GoldSrcServer(randomServer.getAddress(), randomServer.getPort());
	...
\end{lstlisting}

\subsection{Remote controlling a game server (RCON)}
Remote controlling a game server via RCON is done with
the \lstinline{GameServer} objects \lstinline{GoldSrcServer} and
\lstinline{SourceServer}. They feature two methods called
\lstinline{rconAuth()} and \lstinline{rconExec()} used for authenticating with
the server and executing commands on the server respectively.

\begin{lstlisting}[caption=Executing the command \lstinline{status} on a local Source
	server and displaying the output, language=PHP]
	$server = new SourceServer("127.0.0.1");
	$server->rconAuth("password");
	try
	{
		echo $server->rconExec("status");
	}
	catch(RCONNoAuthException $e)
	{
		trigger_error("Could not authenticate with the game server.", E_USER_ERROR);
	}
\end{lstlisting}

RCON communication is completely different in GoldSrc and Source. While GoldSrc
uses the same UDP communication used for queries, Source uses a TCP channel
for RCON requests and responses. Because of this GoldSrc RCON always sends the
password with the command, while for Source a RCON client authenticates once
and is then allowed to execute commands. Therefore authenticating with a wrong
password to a Source server will lead to a \lstinline{RCONNoAuthException}. For
GoldSrc you will not get a \lstinline{RCONNoAuthException} until you try to 
execute a command on the server. This is because GoldSrc RCON always sends the
password with the command.

\section{Using Steam Community features}
A second feature set of \steamcondenser\ can be used to obtain data directly
from the Steam Community. This is done via a XML interface provided by Valve. While
this features is still in progress by Valve, the \steamcondenser\ feature is
also unfinished and very likely to change in future releases.

\subsection{Steam IDs}
The user profiles in the Steam Community are called Steam IDs. The
\steamcondenser\ class representing this profiles is called \lstinline{SteamId}
and is the central class of the Steam Community features, which can be used to
get all of the other information about groups and so on.

\chapter{Reference}
This chapter is all about programming with \steamcondenser\. It includes
descriptions of all classes and their methods in this library and how they work
together.

\section{Server classes}
The objects of the *Serve are the starting point for working with Steam Condensers query functionality.

\subsection{GoldSrcServer / SourceServer}
These classes work pretty similar for the user, but are somewhat different inside.

\subsubsection{bool rconAuth(String password)}
This method excepts a string containing the RCON password for this server. It returns \textbf{true} after successful authentification with the server.

\note{This returns always \textbf{true} for GoldSrcServer objects as GoldSrc uses a pretty simple authentication mechanism. You have to catch RCONNoAuthException when using rconExec if you want to handle failed authentication attemps.}

\subsubsection{String rconExec(String command)}
\label{rconExec}
This method excepts a string containing the command that should be executed on the server via RCON. The returned String is the response from the server just like it would be displayed in the server's console.

\subsection{MasterServer}

\subsubsection{GameServers[] getServers(int regionCode, String filters)}
Returns an array of GameServer objects based on the region code and filters given. If there are no server present, updateServers() is called with the same parameters to receive information about available servers.

\subsubsection{void updateServer(int regionCode, String filters)}
Receives information about available servers from the master server using the given region code and filters.
\chapter{Programming conventions}

\section{Error handling}
Error handling in \steamcondenser\ is purely based on exceptions. At the moment
most of the exceptions are thrown to the user. So you are able to handle them
how you want to. This may change in future releases to provide easier error
handling to the user while remaining flexible.\\
All custom exceptions used are based on a base class called
\lstinline{SteamCondenserException}.

\section{Code style}
Although featuring several programming languages and supporting as much
Steam-related features as possible, \steamcondenser\ also tries to implement a
programming interface both easy to understand and powerful to use. This also
includes matching class and method names between different implementations
assuring easy portable application code. But programming languages are different
and they also feature their very own naming convetions and programming styles.
Therefore it's more or less impossible to have the same names in all
implementations.

\subsection{Class names}
The first and most important part of the isomorphic code base are class names.
These are equal for all important \steamcondenser\ classes. For example the
class \lstinline{SourceServer} is called like that in all implementations. But
Helper classes may differ from this scheme. This is mostly because some
programming languages like Ruby feature a lot of functionality already built
into classes while it has to be build from scratch for less object-oriented
languages like PHP. This results in new classes which don't exist in other
implementations. A good example for this is \lstinline{InetAddress} in PHP.
While a class with similar functionality exists in Java
(\lstinline{InetAddress}) and Ruby (\lstinline{IPAddr}) it had to be
programmed from scratch for PHP. This results in a class unique to the PHP
implementation and is therefore not contained in other implementations.

\subsection{Method names}
Method names are usually given in camel-case (e.g.
\lstinline{SourceServer.getPlayers()}). Ruby's naming conventions demand
underscored method names (\lstinline{SourceServer.get_players()}). All of the
classes of the Ruby implementation include a module
\lstinline{CamelCaseMethods} which rewrites calls to camel-cased methods to
their real counterparts with underscores. This further increases portability
while coding can follow the Ruby code conventions.

\chapter{Known Issues}
\begin{itemize}
  \item The Java implementation \steamcondenser\ is currently not able to reliably handle really long RCON responses. For example the execution of \lstinline{cvarlist} may fail. You should therefore resign from using commands which return long responses (which have to be split over several packets).
\end{itemize}
\chapter{Changelog}

\begin{itemize}
  \item Version 0.7:
  		\begin{itemize}
            \item Added Steam Community features
            \item Fixed detection and decompression of bzip2 compressed packets
            for Java and PHP
            \item Added detection and decompression of bzip2 compressed packets
            for Ruby (requires BZ2 module)
		\end{itemize}
  \item Version 0.6:
  		\begin{itemize}
            \item Added RCON functionality
		\end{itemize}
  \item Version 0.5.1:
  		\begin{itemize}
            \item This version was a bugfix release for the PHP implementation.
		\end{itemize}
  \item Version 0.5:
  		\begin{itemize}
            \item First release of \steamcondenser.
            \item Full query support for GoldSrc and Source game servers implemented in Java, PHP and Ruby.
		\end{itemize}
\end{itemize}
\include{license}

\lstlistoflistings

\end{document}
